\documentclass[12pt,a4paper]{article}
\usepackage[utf8]{inputenc}
\usepackage[czech]{babel}
\usepackage[T1]{fontenc}
\usepackage{amsmath}
\usepackage{amsfonts}
\usepackage{amssymb}
\usepackage{graphicx}
\usepackage{titlesec}
\usepackage[left=2cm,right=2cm,top=2cm,bottom=2cm]{geometry}
\usepackage{indentfirst}
\usepackage{listings}
\usepackage{color}
\usepackage{hyperref}

%Pravidlo pro řádkování
\renewcommand{\baselinestretch}{1.5}

%Pravidlo pro začínání kapitol na novém řádku
\let\oldsection\section
\renewcommand\section{\clearpage\oldsection}

%Formáty písem pro nadpisy (-změněno na bezpatkové \sffamily z původního \normalfont
\titleformat{\section}
{\sffamily\Large\bfseries}{\thesection}{1em}{}
\titleformat{\subsection}
{\sffamily\large\bfseries}{\thesubsection}{1em}{}
\titleformat{\subsubsection}
{\sffamily\normalsize\bfseries}{\thesubsubsection}{1em}{}

%Nastavení zvýrazňování kódu v \lslisting
\definecolor{mygreen}{rgb}{0,0.6,0}
\definecolor{mygray}{rgb}{0.5,0.5,0.5}
\lstset{commentstyle=\color{mygreen},keywordstyle=\color{blue},numberstyle=\tiny\color{mygray}}

\author{Jan Šmejkal}

\begin{document}

%-------------Úvodni strana---------------
\begin{titlepage}

\includegraphics[width=50mm]{img/FAV.jpg}
\\[160 pt]
\centerline{ \Huge \sc KIV/WEB - Webové aplikace}
\centerline{ \huge \sc Semestrální práce }
\\[12 pt]
{\large \sc
\centerline{Rezervační systém pro Centrum Logických Her}
}


{
\vfill 
\parindent=0cm
\textbf{Jméno:} Štěpán Ševčík\\
\textbf{Osobní číslo:} A13B0443P\\
\textbf{E-mail:} kiwi@students.zcu.cz\\
\textbf{Datum:} {\large \today\par} %datum
\textbf{www:} \url{https://logickehry.zcu.cz/}

}

\end{titlepage}

%------------------------------------------

%------------------Obsah-------------------
\newpage
\setcounter{page}{2}
\setcounter{tocdepth}{3}
\tableofcontents
%------------------------------------------

%--------------Text dokumentu--------------
\section{Zadání}
Vytvoření rezervačního systému pro Centrum Logických Her nacházející se ve studovně budovy NTIS. Aplikace umožní uživatelům vytvářet a zobrazovat rezervace deskových her, sledovat dění jednotlivých her pomocí emailových upozornění a inventární správu evidovaných her. ...


{\section{Použité technologie}}
{\subsection{Back-end}}
Jedním z požadavků na aplikaci je systém, ve kterém má fungovat: webový server s PHP. Pro vývoj a produkci byla zvolena verze php 5.6.
{\subsubsection{Kostra aplikace}}
Aplikace funguje na dedikovaném modelu založeném na MVC architektuře. Při každém spuštění se načte odpovídající kontrolér, v němž se vykoná vhodná akce, která zpracuje uživatelův požadavek a připraví výsledek pro šablonovací část aplikace.
{\subsubsection{Přístup k databázi}}
Pro přístup k databázi je použito nativní rozhraní PDO, jehož instance se vytvoří před spuštěním aplikace. Většině tabulek odpovídá modelová třída, v němž se provádí jednotlivé databázové operace.
{\subsubsection{Vytváření a zpracování odkazů}}
Veškerá manipulace s url probíhá v dedikované třídě \textbf{URLGenerator}. Zde lze také nastavit zda-li se mají používat 'hezké url', tedy formát zpracovávané adresy.
{\subsubsection{Šablonování}}
Většině pohledů, neboli stránek či uživatelských akcí, náleží unikátní šablona. Pro zobrazování šablon se využívá knihovna \textit{Twig}, které stačí předat název šablony k zobrazení a data, jež se v šabloně mají zobrazit. Některé šablony pohledů však dědí od jiných šablon, protože mají téměř identickou funkci a je mezi nimi rozdíl v několika drobnostech.
{\subsection{Front-end}}
Kostra front-endu stojí na standardní technologii HTML5, do níž se vzhled vnáší css. Z velké části je kostra také tvořena frameworkem \textit{Twitter Bootstrap} k jehož provozu je také potřeba knihovna \textit{jQuery}.
{\subsubsection{Ovládání aplikace}}
Aplikace se ve většině případů ovládá přímými odkazy mezi jednotlivými pohledy, ačkoliv se v některých částech vyskytují nestandardní ovládací prvky nebo animace, které jsou deklarovány ve specifických .css a .js souborech.

V některých případech pro práci s aplikací ani není třeba přecházet mezi stránkami a jednotlivé akce se provádí pomocí AJAXových požadavků.


{\section{Adresářová struktura}}
Struktura adresářů je navržena pro jednoduchou orientaci a pro zamítnutí přístupu ke zdrojům aplikace zvenčí.

\begin{description}
\item[config]
Specifické konfigurace aplikace, např.: konfigurace databáze, konfigurace jednotlivých modulů
\item[controllers]
Hlavní kontrolér a jednotlivé kontroléry určující strukturu aplikace z pohledu uživatele
\item[libs]
Použité knihovny, moduly konkrétních výkonných logik a takzvaný autoloader, který zajišťuje načtení souborů tříd, které se v běhu aplikace zrovna mají použít.
\item[logs]
Výpisy a hlášení vzniklá při běhu aplikace
\item[model]
Třídy definující datovou část aplikace a třídy, které jsou specifické pro tuto aplikaci
\item[operator]
Složka pro import a export dat z aplikace
\item[templates]
Všechny šablony a části šablon se nachází zde roztříděné podle jednotlivých kontrolérů.
\item[www]
Výchozí složka, kam se návštěvník stránky dostane, obsahuje aplikační zdroje se kterými pracuje prohlížeč jako jsou například styly, skripty nebo obrázky her. Také je zde adresář \textbf{webauth}, pomocí kterého funguje přihlašování pomocí jednotného orion loginu.
\end{description}
%------------------------------------------

\end{document}